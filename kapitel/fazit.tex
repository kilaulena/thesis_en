\chapter{Evaluation and future prospects}
\label{chap:fazit}

In order to evaluate the result, this chapter will compare the system that was developed with the requirements set in chapter \ref{chap:systemanforderungen}. The result will be evaluated with the aid of these requirements and any problems left unsolved will be discussed. Next, some future prospects for the technologies that were used will be examined. Finally, some recommendations for further development and research will be uttered.

\section{Evaluation of the result}

All seventeen must-haves (see section \ref{subsec:muss}) have been completely implemented. Of the sixteen may-haves however (see section \ref{subsec:kann}), only four could be realised. For want of time, the moving, deleting, commenting and versioning of lines in the outliner have been omitted, as well as selective replication of a single outline and the explicit enabling and disabling of replication.

In order to create a truly usable product, first of all it would be necessary to further improve conflict handling. As described in section \ref{subsec:otherconflicts-design}, conflicts caused by simultaneous and diverging indenting and outdenting of lines will lead to errors in the outline's layout. If lines are simultaneously changed in three or more replicas this will also lead to conflicts that are not correctly dealt with. This is chiefly an interaction problem: the user interface for manual conflict resolution, as described in section \ref{subsec:writeconflict-implementierung}, was built to show only two versions of a line at a time. Though unlikely to occur in everyday use, manually merging a higher number of line versions will necessitate another user interface approach.

Non-functional requirements were discussed to the greatest possible extent. The source code may be improved in simplicity and redundancy: in order to make the program truly compliant to the MVC architecture, the functions that control replication should be rewritten in object-oriented programming style. This was already done for conflict recognition, presentation and handling. The application also meets the requirements of the user interface (cf. section \ref{subsec:gui-anf}). However, due to the necessary connection establishment between front-end and database, the user interface is sometimes a little slow in its reaction to user input, interrupting the user's flow of work for a few seconds. It should be verified if this can be improved by optimising the user interface.

Notwithstanding its shortcomings, the application as described in the project definition was by all means successfully developed and realised. This thesis' most important accomplishment is that the application puts a certain paradigm of Internet use into practice: peer-to-peer communication. Compared to conventional client-server applications, the architecture conceived here allows users to better control their data. On-line collaboration was implemented without depending on uninterrupted Internet access and ever-available servers.


\section{The future of the employed technologies}

The implementation was slower than imagined. In order to aptly employ the HTTP API, replication, conflict handling and monitoring the database for changes a lot of background work had to be done first. Comparatively little time was left for the actual writing of business logic. Yet, the technologies used were continuously improved and new libraries and frameworks were developed that will reduce the amount of work for similar projects in the future.

For instance, CouchDB 1.0 contains some features that may definitely make it easier to improve the application \cite{couch:whatsnew}. Its release is due when this thesis is finished \cite{couch:release1.0}. CouchDB version 1.0 will among other things allow individual document replication using a document's ID, eliminating the need to replicate the database as a whole. This makes it easier to perform selective replication of outlines. Furthermore, the support for Windows operating systems has been improved, further increasing its platform-independency. Future CouchDB releases will also natively support sharding [Lehnardt, Jan, personal conversation, 9 July 2010]. This will eliminate the need for CouchDB Lounge in the future.

Among recent developments, the framework \textit{Evently} deserves special notice \cite{evently:website}. Like Sammy, Evently allows application routing, but it was specially designed with event-based CouchDB applications in mind. Evently creates a connection between CouchDB views, the changes feed, HTML templates and any defined JavaScript callbacks, and gives a structure for the organisation of the source code. Compared to the means used in this thesis, Evently will certainly increase productivity.


\section{Suggestions for further development}

The outliner can be developed further without major hindrances. The use of the tree structure will make it easy to implement deleting and moving of lines by no longer displaying the line or fitting it in somewhere else in the tree. Columns in the outliner can also effortlessly be implemented by assigning several text areas to the lines.

Another field of work would be the implementation of access control and user administration. Individual outlines could be marked as private or public by users and therefore available for replication or not. Applications that use distributed data on mobile devices require higher security measures:

\begin{quote}
Providing high availability and the ability to share data despite the weak connectivity of mobile computing raises the problem of trusting replicated data servers that may be corrupt. This is because servers must be run on portable computers, and these machines are less secure and thus less trustworthy than those traditionally used to run servers. [...] Portable machines are often left unattended in unsecured or poorly secured places, allowing attackers with physical access to modify the data and programs on such computers. \citelit[Chap. 1]{servercorruptness}
\end{quote}

Accordingly, if this application is to be developed for productive use, implementation of access control should be highly prioritised. Yet, the author of this thesis believes that the highest priority should be to extend the application's peer-to-peer capability. Application instances could propagate approved outlines through a web service. Protocols such as \textit{Bonjour} make it possible to detect network services in local IP networks \cite{bonjour:website}. Such a protocol could be used so that application instances might recognise each other inside a network and offer the possibility to replicate outlines directly with one another. In doing so, documents could be co-edited simultaneously even without an Internet connection, for instance inside an office or on conferences.
 
The implementation of the tasks was certainly successful. However, a considerable amount of development effort is still needed in order to make the application truly suitable for the uses listed in section \ref{subsec:einsatzmoegl}. If several users should massively and simultaneously edit an outline -synchronising only after many changes have been made- the number and complexity of conflicts cannot yet be dealt with in a satisfying and stable manner. It is certainly possible to implement a distributed system using CouchDB and a selection of other technologies, but data merging remains the application developer's responsibility. The CouchDB development team however plans to provide built-in solutions to standard conflict resolution scenarios in the future [Lehnardt, Jan, personal conversation, 9 July 2010]. That said, merging is exceptionally difficult in the use case at hand, since the documents (i.e. the lines in an outline) are very granularly chosen and strongly linked together. For application areas where documents are not often edited simultaneously and therefore cause less conflicts, the solution may be much less complex. Conceivable cases are address books, calendars, customer data, or even services that exchange messages. \citelit[Chap. 10]{couchdb} and \cite{couch:whatsnew} contain further suggestions.


